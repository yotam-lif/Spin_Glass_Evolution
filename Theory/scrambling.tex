%% LyX 2.4.1 created this file.  For more info, see https://www.lyx.org/.
%% Do not edit unless you really know what you are doing.
\documentclass[american]{article}
\usepackage[T1]{fontenc}
\usepackage[utf8]{inputenc}
\usepackage{graphicx}
\usepackage{babel}
\begin{document}
\title{Scrambling in the Spin-Glass Evolution Simulation}

\maketitle
Observe the ``scrambling'' effect in the simulation.

Say we define some initial time $t_{0}$ and time variable $t$ that
starts from $t_{0}$.

Remember $\vec{\alpha}$ is the binary genome vector, $\hat{J}$ is
the epistatic matrix.

\[
F(\vec{\alpha})=\sum_{i=1}^{L}h_{i}\alpha_{i}+\sum_{i,j=1}^{L}\alpha_{i}J_{ij}\alpha_{j}
\]

And the fitness effect of flipping $\alpha_{i}\rightarrow-\alpha_{i}$:

\[
\Delta_{i}=-2\alpha_{i}\left(h_{i}+\sum_{j=1}^{L}J_{ij}\alpha_{j}\right)
\]

Define the local field $f_{i}$:

\[
f_{i}:=\sum_{j=1}^{L}J_{ij}\alpha_{j}
\]

fig:t_0=1000001
\[
F(\vec{\alpha})=\sum_{i=1}^{L}(h_{i}+f_{i})\alpha_{i}
\]

\[
\Delta_{i}=-2\alpha_{i}\left(h_{i}+f_{i}\right)\approx-2\alpha_{i}f_{i}
\]

Where we disregard the $h_{i}$ for this effect, as we assume it happens
for $\beta\approx1$.

Define ``Forward DFE'' or ``Forward propagation'' the act of taking
the beneficial DFE (BDFE) at time $t_{0}$, and observing the genes
responsible for this distribution (these would be the genes that have
$\alpha_{i}f_{i}<0$). Propagate the system to time $t$, and plot
the distribution of fitness effects of these same genes. Are they
still all beneficial, or perhaps the distribution changes?

The same can be defined as ``Backward DFE'' or ``Backward propagation''.
Take the BDFE at time $t$ and propagate the system back to time $t_{0}$,
and observe the distribution of fitness effects of the genes responsible
for the BDFE at $t_{0}$. 

We plot the propagations for $t_{0}=0$ (\ref{fig:t_0=00003D0}) and
for $t_{0}=1+10^{6}$ (\ref{fig:t_0=00003D1000001}). 
\begin{figure}
\begin{centering}
\includegraphics[scale=0.4]{/Users/yotamlif/Desktop/Spin_Glass_Evolution/dfe_tracker_plots_lenski_data_10-6/replicate0/strain_lineage_0/dfe_day_0}
\par\end{centering}
\begin{centering}
\includegraphics[scale=0.4]{/Users/yotamlif/Desktop/Spin_Glass_Evolution/dfe_tracker_plots_lenski_data_10-6/replicate0/strain_lineage_0/dfe_day_200000}
\par\end{centering}
\begin{centering}
\includegraphics[scale=0.4]{/Users/yotamlif/Desktop/Spin_Glass_Evolution/dfe_tracker_plots_lenski_data_10-6/replicate0/strain_lineage_0/dfe_day_3000000}
\par\end{centering}
\caption{ Forward and backward propagations for $t_{0}=0,$ $t=[0,2\cdot10^{5},3\cdot10^{6}].$
We can observe the slow evolution of the propagated DFEs into what
seems to be a gaussian.}\label{fig:t_0=00003D0}

\end{figure}

\begin{figure}
\begin{centering}
\includegraphics[scale=0.4]{/Users/yotamlif/Desktop/Spin_Glass_Evolution/dfe_tracker_plots_lenski_data_10-6/replicate0/strain_lineage_0/dfe_day_1000001}
\par\end{centering}
\begin{centering}
\includegraphics[scale=0.4]{/Users/yotamlif/Desktop/Spin_Glass_Evolution/dfe_tracker_plots_lenski_data_10-6/replicate0/strain_lineage_0/dfe_day_1500001}
\par\end{centering}
\begin{centering}
\includegraphics[scale=0.4]{/Users/yotamlif/Desktop/Spin_Glass_Evolution/dfe_tracker_plots_lenski_data_10-6/replicate0/strain_lineage_0/dfe_day_4000001}
\par\end{centering}
\caption{ Forward and backward propagations for $t_{0}=1+10^{6},$ $t=[1+10^{6},1+15\cdot10^{5},1+4\cdot10^{6}].$
We can observe the slow evolution of the propagated DFEs into what
seems to be a gaussian. Timescale is larger than before because of
diminishing magnitude of incorporated fitness effects, leading to
larger fixation times.}\label{fig:t_0=00003D1000001}
\end{figure}
So it seems the value of $t_{0}$ doesn't change this effect. We shall
immediately attempt to analyze this effect, but beforehand we must
state the obvious -

It must be that this Gaussian is a transitionary effect, and such
is the one observed in Bayms paper as well. I say this because it
is obvious that the BDFE loses it's meaning / distribution nature
when we approach a local fitness peak and there are almost no beneficial
distribution effects left. Thus, we must think of this creature in
the right regimes of existence.

Now, to analyze:

\[
\Delta_{i}=-2\alpha_{i}\left(h_{i}+f_{i}\right)\approx-2\alpha_{i}f_{i}
\]

Define $N_{+}(t)$ as the number of $\alpha_{i}=+1$ at time $t$
and $N_{-}(t)=L-N_{+}(t)$ the number of $\alpha_{i}=-1.$

$\Delta_{i}$ are the fitness effects. If we look at the BDFE, we
are \textbf{conditioning on $i$ s.t} $\Delta_{i}>0$, i.e. $\alpha_{i}f_{i}<0$.
Observe that in general and regardless of time there exists a symmetry
of $J_{ij},h_{i}$ in the system around $0$, thus we assume $N_{+}(t)\approx N_{-}(t)$
regardless of $t$.

The only thing that changes is that $\alpha_{i}$ flip to match their
$f_{i}$. But, when we condition on $\Delta_{i}>0$ we choose the
$i$s s.t their $f_{i}$ are the ones that this symmetry is broken
in a way that $f_{i}\neq0$ , and in particular $sign(f_{i})\neq sign(\alpha_{i}).$ 

For instance, if $\alpha_{i}<0$ and the fitness effect of gene $i$
is beneficial, this means $f_{i}=\sum_{j=1}^{L}J_{ij}\alpha_{j}>0$
and in particular, considering $N_{+}(t)\approx N_{-}(t)$, it must
be that specifically for row $i$ we have an untypical amount of $J_{ij}$
that have the same sign as $\alpha_{j}$.

Evolving in time, we assume this randomly flips $\alpha_{j}$. We
had more terms of $J_{ij}\alpha_{j}>0$ than $J_{ij}\alpha_{j}<0$
for the above example, such that flipping $\alpha_{j}$ more likely
evens out the sum such that there roughly the same number of $J_{ij}\alpha_{j}>0$
as $J_{ij}\alpha_{j}<0$.

Observe that now, for large $L\rho$ and an unbiased sum:

\[
f_{i}:=\sum_{j=1}^{L}J_{ij}\alpha_{j}\approx\sum_{j=1}^{L}J_{ij}\rightarrow X_{i}\sim N(0,\sigma_{J}\sqrt{L\rho})
\]

Where we assumed that for the now unbiased sum, randomly flipping
the signs of $J_{ij}$ does not change the result because they are
drawn from a normal distribution. 

So we assume that from the CLT, $f_{i}$ becomes a normally distributed
RV. 

Because we have:

\[
\Delta_{i}\approx-2\alpha_{i}f_{i}
\]

The other terms can, at most, flip the sign of $f_{i}$. So we see
that the DFE is the distribution of $f_{i}$s who are perhaps flipped,
which does not change the distribution as they are normally distributed
RVs.

This can explain perhaps the ``scrambling'' effect we observe in
the simulation, and that which is observed in Bayms paper.
\end{document}
